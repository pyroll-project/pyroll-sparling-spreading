\documentclass[11pt]{PyRollDocs}
\usepackage{textcomp}

\addbibresource{refs.bib}

% Document
\begin{document}

    \title{The Sparling Spreading PyRoll Plugin}
    \author{Christoph Renzing}
    \date{\today}

    \maketitle

    This plugin provides a spreading modelling approach with Sparling's formula for flat rolling, adapted on groove rolling by an equivalent rectangle approach.


    \section{Model approach}\label{sec:model-approach}

    \subsection{Sparling's spread equation}\label{subsec:sparling's-spread-equation}

    \textcite{Sparling1961} proposed \autoref{eq:sparling} for estimation of spreading in flat rolling,
    $h$ and $b$ are height and width of the workpiece with the indices 0 and 1 denoting the incoming respectively the outgoing profile.
    $a$, $b$, $f$, $g$ and $j$ are correction coefficients for roll surface, bar surface, material, temperature and strain rate, respectively.

    \begin{equation}
        \beta = \frac{b_1}{b_0} =  \frac{h_0}{h_1} ^{w a b f g j}
        \label{eq:sparling}
    \end{equation}


    $w$ is the spread exponent and given by \textcite{Sparling1961} is given in \autoref{eq:exponent}, where $R$ is the roll radius.

    \begin{equation}
        w = 0.981 \exp \left[ -0.6735 \left( \frac{2.395 b_0^{0.9}}{R^{0.55} h_0^{0.1} \Delta h^{0.25} } \right) \right]
        \label{eq:exponent}
    \end{equation}


    \section{Usage instructions}\label{sec:usage-instructions}

    The plugin can be loaded under the name \texttt{pyroll.sparling\_spreading}.

    An implementation of the \lstinline{spread} hook on \lstinline{RollPass} is provided,
    calculating the spread using the equivalent rectangle approach and Sparling's model.
    Also, an implementation of the \lstinline{pillar_spreads} hook on \lstinline{RollPass.DiskElement} is provided,
    for using the spread equation with the pillar model approach.
    This implementation is only provided, if \texttt{pyroll.pillar\_model} is installed.

    Several additional hooks on \lstinline{RollPass} are defined, which are used in spread calculation, as listed in \autoref{tab:hookspecs}.
    Base implementations of them are provided, so it should work out of the box.
    For \lstinline{sparling_exponent} the equations~\ref{eq:exponent} is implemented.
    The others default to \num{1}.
    Provide your own hook implementations or set attributes on the \lstinline{RollPass} instances to alter the spreading behavior.

    \begin{table}
        \centering
        \caption{Hooks specified by this plugin. Symbols as in \autoref{eq:sparling}.}
        \label{tab:hookspecs}
        \begin{tabular}{ll}
            \toprule
            Hook name                                     & Meaning                                 \\
            \midrule
            \texttt{sparling\_roll\_surface\_coefficient} & roll surface correction coefficient $a$ \\
            \texttt{sparling\_bar\_surface\_coefficient}  & bar surface correction coefficient $b$  \\
            \texttt{sparling\_material\_coefficient}      & material correction coefficient $f$     \\
            \texttt{sparling\_temperature\_coefficient}      & temperature correction coefficient $g$  \\
            \texttt{sparling\_strain\_rate\_coefficient}      & strain rate correction coefficient $j$  \\
            \texttt{sparling\_exponent}                   & spread exponent $w$                     \\
            \bottomrule
        \end{tabular}
    \end{table}

    \printbibliography

\end{document}